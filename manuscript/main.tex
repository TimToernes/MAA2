%\documentclass[11pt,a4paper,english,twocolumn]{article}
\documentclass[11pt,a4paper,english]{article}
% Graphics packages 
\usepackage{graphicx}
% Include the Euro character
\usepackage[utf8]{inputenc}
\usepackage{lmodern,textcomp}

% Line numbering
%\usepackage[switch,columnwise]{lineno}
%\linenumbers

% Math Package
\usepackage{mathtools}
\usepackage{amsmath}
% Bibliography package 
\usepackage[backend=bibtex, style=numeric,citestyle=numeric,sorting=none, backref=none, backrefstyle=all+]{biblatex}
\addbibresource{references.bib}

\begin{document}
	
The goal of this method is to evenly sample the near-optimal feasible decision space of a linear optimization model using a version of Markov Chain Monte Carlo (MCMC) sampling named the hit and run algorithm \cite{chen_fast_nodate, kiatsupaibul_analysis_2011}.
	
\section{Model}

The model under consideration is linear and convex and can be written on the standard form: 

\begin{align}\label{eq:ConvexOptimization}
\text{minimize} \;&\; \mathbf{f}_0(\mathbf{x})  \\
	\text{subject to} \; &\; \mathbf{f}_i(\mathbf{x}) \leq 0 \; \; i=1...m\\
\;            &\;  \mathbf{h}_j(\mathbf{x}) = 0 \; \; j=1...p
\end{align}

With $\mathbf{x}\in \mathbf{R}^n$. Besides from the constraints contained in the original optimization problem, an additional constraint is introduced, to separate the near-optimal feasible space, from the entire feasible space. 
\begin{equation}\label{eq:MGA_const}
f_0(\mathbf{x}) \leq f_0(\mathbf{\hat{x}})\cdot (1+\epsilon)
\end{equation}

Where $\mathbf{\hat{x}}$ is an optimal solution to the original problem and $f_0(\mathbf{\hat{x}})$ is the objective value for the optimal solution. The slack on optimality is given by the slack variable $\theta$ .

As the constraints are linear they can be written as matrix products on the form: 

\begin{align}
\mathbf{f}_i(\mathbf{x}) \leq 0 \Rightarrow A  \mathbf{x}  \leq \mathbf{b} \\
\mathbf{h}_j(\mathbf{x}) = 0 \Rightarrow H  \mathbf{x} = \mathbf{c}
\end{align}

The set containing all feasible near-optimal solutions then become:
\begin{equation}
F = \{\mathbf{x} \in \mathbf{R}^n | A  \mathbf{x}  \leq \mathbf{b} \wedge H  \mathbf{x} = \mathbf{c} \}
\end{equation}


\section{Presolve}

In order to apply the hit and run sampling algorithm, a fully dimensional space is required. As several equalities are included in the definition of $F$ and hidden equalities might be found among the inequality constraints, the fully dimensional subspace of $F$ must be found. Hidden equalities occur when two inequalities limit the range of a variable to zero. 

The goal of the presolve process is to define the fully dimensional subspace $Z$ of $F$. 

\subsection{Step 1}
Initially all linearly dependent constraints/rows in the augmented matrix $[ A|\mathbf{b}] $ are identified. Linearly dependent constraints/rows, span parallel hyper-planes in $F$. If two such hyper-planes were to coincide, and have normal vectors pointing in opposite directions, they constrain a dimension of $F$ and can be represented as an equality rather than two inequalities. For all such inequality constraints constraining a dimension of $F$, their given row in $A$ are moved from the matrix $A$ to the $H$ matrix.

\subsection{Step 2}
From \cite{ConvexOpimization} (p. 523) we know that the problem can be reformulated as: 

\begin{align}
\text{minimize} \;&\; \mathbf{f}_0(\mathbf{\hat{x}} + N \mathbf{z})  \\
\text{subject to} \; &\; A(\mathbf{\hat{x}} + N \mathbf{z}) \leq \mathbf{b} \; \; i=1...m\\
\;            &\;  span(N) = \mathcal{N}(H)
\end{align}

Where the span of $N$ is the null space of $H$ and $\mathbf{\hat{x}}$ is any particular solution contained in $F$.
As all equalities have been eliminated from the problem, the fully dimensional subspace of $F$ can be defined as:

\begin{equation}
Z = \{\mathbf{z} \in \mathbf{R}^{n-p} | A(\mathbf{\hat{x}} +  N \mathbf{z}) <= \mathbf{b}  \}
\end{equation}

Using this the optimization problem can be reformulated as:

\begin{align}
	\text{minimize} \;&\; \mathbf{f}_0(\mathbf{\hat{x}} + N \mathbf{z})  \\
	\text{subject to} \; &\; \hat{A}\mathbf{z} \leq \mathbf{\hat{b}} \label{eq:sub_problem}
\end{align}

With $\hat{A}$ and $ \mathbf{\hat{b}}$:

\begin{align}
	\;            &\;  \hat{A} = A \cdot N \\ 
	\;			& \; \mathbf{\hat{b}} = \mathbf{b}- A\mathbf{\hat{x}}
\end{align}

By validating that the matrix $\hat{A}$ has full rank, we know that the subspace $Z$ is fully dimensional. 

\subsection{Calculating the Null space $N$}

Using QR decomposition the Null space of a matrix can be calculated. 

Given the $(m \times n)$matrix $H$, it can be decomposed to the orthogonal $(m\times m)$ matrix Q, and R which is a $(m \times n)$ upper triangular matrix. 

\begin{equation}
	H = QR = \left[Q_1  Q_2 \right] \left[{R_1 \atop 0} \right]
\end{equation}

Where $Q_1$ is $(m \times r )$ and $Q_2$ $(m\times m-r)$, with $r$ being the rank of $H$. 

The null space of $H$ is can then by found as $Q_2 = N(H^T)$

For proof see \cite{Hyde}
\section{Hit and run sampling}

A version of the hit and run algorithm \cite{Smith1984}, is used to sample the subspace $Z$. The algorithm consist of an iterative process of drawing random directions, and taking steps of random lengths within the space $Z$. 

Initially a point $\mathbf{z}_0$ inside the space $Z$ must be provided. Random steps inside Z are then taken by generating a random direction $\theta$ and taking a random step $t$ that will not violate the boundaries of $Z$. 

\begin{equation}\label{eq:step}
	\mathbf{z}_{i+1} = \mathbf{z}_i + \theta t
\end{equation}

%todo update this 
$\theta$ must be a unit vector pointing in a random direction evenly distributed on a unit hyper sphere $\mathbf{S}^{n-p}$. I will elaborate on how to do so!!! But this is easy.

To determine the range of $t$ that ensures that the step $t$ does not cross the boundary of $Z$, Equation \ref{eq:step} is substituded in to Equation \ref{eq:sub_problem}.

\begin{equation}
	\hat{A}(\mathbf{z}_i + \theta t) \leq \hat{b}
\end{equation}

Isolating $t$ will result in the upper and lower bounds:

\begin{align}
t_{max} =  \text{inf} \frac{\hat{b}-\hat{A}\mathbf{z}_i}{\hat{A}\theta}& \; \text{given }  \; \hat{A}\theta>0\\
t_{min} = \text{sup} \frac{\hat{b}-\hat{A}\mathbf{z}_i}{\hat{A}\theta}& \; \text{given }  \; \hat{A}\theta<0
\end{align}

Selecting $t$ to be:

\begin{equation}
	t = (t_{max}-t_{min})r+t_{min}
\end{equation}

Where $r$ is a random number drawn from a uniform distribution between 0-1. 

Repeating the process of generating random directions $\theta$ and taking random step lengths $t$, will, if enough samples are drawn, generate a uniform sampling of the near optimal feasible set $Z$. Storing all samples $z_i$ in the discrete set $Z^* = {z_i \forall i=0..d}$ where $d$ is the number of samples.\\


"The reason that the Markov chain corresponding to the iterates $Z_0,Z_1,...,Z_d$ converges in distribution to a uniform distribution over $Z$ is easily seen from the fact that 1) it is possible to go from any point in $Z$ to a neighborhood of any other point in one step, and 2 ) the uniform distribution is a stationary distribution of the chain." \cite{Smith1996}\\


The hit and run algorithm is $md$ hard \cite{Belisle1998}. Where m is the number of constraints and d is the number of variables. 

\section{Decrush}

Having sampled $Z$, these sample points must be reverted back to the $F$ domain. This is done with a so called decrush algorithm. 

\begin{equation}
	\mathbf{x}_i = N \mathbf{z}_i + \mathbf{\hat{x}}
\end{equation}

Repeating this for all samples a set containing $d$ samples in $F$ is obtained. 

\section{Further work}

Figure out if sampling $Z$ evenly, generates evenly distributed samples in $F$ when samples are maped from $Z$ to $F$. \\

As of now, i have had sucess with applying the method on problems with $n\approx 10.000$, and i believe that $n\approx 100.000$ is doable on a better pc. This is however one to two order of magnitude from the $1.000.000+$ variables included in most modern energy system optimization models. Therefore i have considered two options. 1) Using decomposition methods, such as Benders decomposition to decompose the problem in to manageable sub-problems. Specifically it is only the non-timedependable variables that are truly of interest, and as these only make up $\approx 1000$ of the variables in the energy system model. By decomposing the timedependable variables from the non-timedependable variables, and sampling the non-timedependable subspace, could be a solution. I have, however, not managed to figure out if this is a viable solution, as decomposition methods often require several iterations between the two sub-problems.  
2) The largest contributor to the large number of variables in energy system models, is the time-series data used. By generating augmented time-series, using techniques such as self organizing maps \cite{Hasan2019}, one can generate shorter time-series, thereby reducing the problem. This is however, a less favorable option than solution (1) as using augmented time-series 

\clearpage
\printbibliography

\end{document}